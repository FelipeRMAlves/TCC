\chapter{Revisão Bibliográfica}

Em um disco de freios ocorrem os três tipos de transferência de calor descrito por  Frank P. Incropera ~\cite{Lan50a}: condução, convecção e radiação. O autor define transferência de calor como energia térmica em trânsito devido a uma diferença de temperatura no espaço. Quando existe um gradiente de temperatura em um meio estacionário sólido ou fluido a condução ocorre quando há transferência de calor através do meio. Já a convecção ocorre quando há trasferência de calor entre uma superfície e um fluido em movimento. Quando a troca de calor ocorre através de ondas eletromagnéticas que são emitidas por qualquer superfície com temperatura não nula, ocorre a radiação. O estudo desses fenômenos ocorre através de equações de taxas que quantificam a energia transferida por unidade de tempo.

A condução pode ser interpretada como um fenômeno difusivo onde ocorre transferência de energia das partículas mais energéticas para as menos energéticas através das interações entre partículas. Em um sólido essas interações ocorrem através da combinação entre a vibração das moléculas dos retículos cristalinos e a movimentação dos elétrons livres. Para a condução a equação que descreve a taxa de transferência de calor é conhecida como a \textit{lei de Fourier}.

$$ \textbf{q} = -k \nabla \textbf{T} = -k \left(\textbf{i} \frac{dT}{dx} + \textbf{j} \frac{dT}{dy} + \textbf{k} \frac{dT}{dz} \right) $$

sendo $\textbf{q}$ o fluxo de calor e $k$ a condutividade térmica.



Denotemos o domínio por $\Omega$\symbl{$\Omega$}{domínio de definição de uma
equação diferencial}. Seja $\partial \Omega$ o contorno de
$\Omega$.\symbl{$\partial$}{operador do contorno.}

\begin{equation}
	|x| = \left\{ \begin{array}{ll}
	1 & \mbox{, se } x \geq 0; \\
	-1 & \mbox{, se } x < 0. \end{array} \right.
\end{equation}

